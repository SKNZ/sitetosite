\documentclass[paper=a4, fontsize=11pt]{scrartcl}

\usepackage[utf8]{inputenc}
\usepackage{fourier} % Adobe Utopia
\usepackage[english]{babel}
\usepackage{amsmath,amsfonts,amsthm}
\usepackage{relsize}
\usepackage{svg}

\usepackage{sectsty}
\allsectionsfont{\normalfont\scshape} 

\usepackage{fancyhdr}
\pagestyle{fancyplain}
\fancyhead{}
\fancyfoot[L]{} 
\fancyfoot[C]{}
\fancyfoot[R]{}
\renewcommand{\headrulewidth}{0pt}
\renewcommand{\footrulewidth}{0pt}
\setlength{\headheight}{14.6pt}

\usepackage{stmaryrd}
\usepackage{url}
\usepackage{blindtext}
\usepackage{enumitem}

\hoffset = -40pt
\voffset = -40pt
\textwidth = 500pt
\textheight = 760pt

\author{Floran NARENJI-SHESHKALANI \& Jean-Marcellin TRUONG} 

\begin{document}

Authors: Floran NARENJI-SHESHKALANI (643166) \& Jean-Marcellin TRUONG (643357)

\section{General idea}

The original design is composed of several IOT clients exchanging with a single
server on the same local area network (called IOT LAN further on).
The stated objective is to move that server to the cloud, and make it so that
this server can be scaled up while keeping the communications secure without
usage of HTTPS\@.
The servers in the cloud will have their own local area network (called cloud
LAN further on).
While cloud servers would normally be accessible over the internet, we will
assume in this case that neither LAN network is connected to the Internet.

To enable safe communications, we will setup a site-to-site VPN over the
Internet which will permit both sites to communicate with one another.

For our scalability goal, we will set up a reverse proxy that will serve as an
endpoint for our IOT connections and will balance them between various back-end
servers.

\section{Proof of Concept limitations}

\subsection{Docker}

While the scenario intends for this setup to be used in a corporate environment,
we have for obvious reasons to simulate such an environment.
For that purpose, we chose to use Docker containers, which allow us a quick and
easy way to setup multiple simulated machines and to script them to our needs.
The main advantage of Docker over traditional virtual machines is of course
reproducibility.

\subsection{IOT device \& server}

The specifications state the limitations of the IOT device but nothing about
it's actual functions.
Therefore, believing this to be of no interest, we have created a simple python
script that is running inside its own Docker container as a way to simulate the
IOT client.
This script simply sends a plain-text HTTP GET request to a specified IP
address/port containing the current date.

Just the same, the server is a simple stateless python script running inside its
own container.
The server receives the data from the client over HTTP, prints it out and does
nothing more with it.
This makes the server very easy to scale, but it also serves no obvious
purpose.

\section{Technical choices}

\subsection{VPN}

We considered various options for the VPN software, but it finally boiled down
to picking between OpenVPN and Strongswan.
It seemed that both softwares could satisfy our usecases and seemed equally
strong and safe.
However, Strongswan implements IPSec, which seems to be the industry standard.
Furthermore, we already had past experience setting up OpenVPN on our own
dedicated servers.
Thus, we elected to use Strongswan as we felt it was a good opportunity to try
it out.

\subsubsection{Authentication}

For authentication, we initially wanted to set up our own public key
infrastructure such that we would have a root certificate and one certificate
for each of our gateways.
Unfortunately, as of now, we have not yet managed to get it working and
authentication systematically fails, even though we believe all the certificates
are signed adequately.

Consequently, we've fallen back onto onto Pre-Shared Key (PSK) authentication,
where the password is stored in clear-text in the Strongswan's configuration
file.
Because our VPN usage scenario is so simple (there will be at most two
participants in the VPN), we believe this to be as easily maintained as a
PKI-based authentication solution.\\

In terms of security, we believe both solutions to be equivalent in most
regards: if the machine is compromised, then a PSK or a private key can be
extracted in the same manner, and both can then be replaced easily by an admin. 
If the private key is password protected, Strongswan will ask for it on startup,
but the same can also be applied to PSK\@.
For an ongoing connection, having access to the original mean of authentication
is useless as it is only used for initial authentication and not as a session
key.\\

However, a PKI would still be best for multiple reasons.
First of all, an existing organization is likely to have its own PKI already,
and it would be best to leverage such an existing system instead of ``rolling
one's own'' through a PSK\@.
Then, while our current needs are quite simple and are easily satisfied by a
PSK, our future might not be so: we might have more sites, more clients and
potentially a need for revocation.
Such a change in need is clearly a usecase of the PKI\@: it provides a tremendous
advantage in scalability and maintenance.
Finally, a key provided through a PKI will hopefully always be fairly long,
whereas a PSK chosen by sloppy students might only be 7 characters long (where
it should be a very long random string).

\subsubsection{Communications security}

Strongswan accepts 3 paramaters related to communications security.
% For encryption, we picked chacha20 combined with poly1305 for MAC\@.
For encryption, we picked AES256-CBC for its MAC properties\@.
% This is equivalent to AES but is apparently less prone to timing attacks.

For integrity, we picked prfsha512, which is sha512 combined with a pseudorandom
function.

Finally, we use elliptic curve based Diffie-Hellman with curve ecp512bp.

\section{Questions}

Can we assume that the LAN network is safe? (e.g.\ do we need to add additional
security to the requests travelling from the reverse proxy to the server)?\\

Is it fine if we have to update the IP\@?
If we don't want to, what solution would be recommended? Can we have the
gateway masquerade as the ``old'' server and have it route the actual requests to
the reverse proxy (NAT?)?\\

Do we need to give an actual purpose to our IOT client/server (i.e.\ such that
it isn't a stateless everything)?

\end{document}

